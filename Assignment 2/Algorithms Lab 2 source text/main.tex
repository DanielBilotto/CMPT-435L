%%%%%%%%%%%%%%%%%%%%%%%%%%%%%%%%%%%%%%%%%
%
% CMPT 435
% Lab Zero
%
%%%%%%%%%%%%%%%%%%%%%%%%%%%%%%%%%%%%%%%%%

%%%%%%%%%%%%%%%%%%%%%%%%%%%%%%%%%%%%%%%%%
% Short Sectioned Assignment
% LaTeX Template
% Version 1.0 (5/5/12)
%
% This template has been downloaded from: http://www.LaTeXTemplates.com
% Original author: % Frits Wenneker (http://www.howtotex.com)
% License: CC BY-NC-SA 3.0 (http://creativecommons.org/licenses/by-nc-sa/3.0/)
% Modified by Daniel Bilotto
%
%%%%%%%%%%%%%%%%%%%%%%%%%%%%%%%%%%%%%%%%%

%----------------------------------------------------------------------------------------
%	PACKAGES AND OTHER DOCUMENT CONFIGURATIONS
%----------------------------------------------------------------------------------------

\documentclass[letterpaper, 10pt]{article} 

\usepackage[english]{babel} % English language/hyphenation
\usepackage{graphicx}
\usepackage[lined,linesnumbered,commentsnumbered]{algorithm2e}
\usepackage{listings}
\usepackage{fancyhdr} % Custom headers and footers
\pagestyle{fancyplain} % Makes all pages in the document conform to the custom headers and footers
\usepackage{lastpage}
\usepackage{url}


\fancyhead{} % No page header - if you want one, create it in the same way as the footers below
\fancyfoot[L]{} % Empty left footer
\fancyfoot[C]{page \thepage\ of \pageref{LastPage}} % Page numbering for center footer
\fancyfoot[R]{}

\renewcommand{\headrulewidth}{0pt} % Remove header underlines
\renewcommand{\footrulewidth}{0pt} % Remove footer underlines
\setlength{\headheight}{13.6pt} % Customize the height of the header

%----------------------------------------------------------------------------------------
%	TITLE SECTION
%----------------------------------------------------------------------------------------

\newcommand{\horrule}[1]{\rule{\linewidth}{#1}} % Create horizontal rule command with 1 argument of height

\title{	
   \normalfont \normalsize 
   \textsc{CMPT 435 - Fall 2021 - Daniel Bilotto} \\[10pt] % Header stuff.
   \horrule{0.5pt} \\[0.25cm] 	% Top horizontal rule
   \huge Assignment Two -- \LaTeX ~Sorts \\     	    % Assignment title
   \horrule{0.5pt} \\[0.25cm] 	% Bottom horizontal rule
}

\author{Daniel Bilotto \\ \normalsize danielbilotto1@marist.edu}

\date{\normalsize\today} 	% Today's date.

\begin{document}

\maketitle % Print the title

%----------------------------------------------------------------------------------------
%   CONTENT SECTION
%----------------------------------------------------------------------------------------

% - -- -  - -- -  - -- -  -
\section{Main Method}.
\begin{lstlisting}[language = java]
//this main method prints out the number of comparisons from all the sorts
//it reads in the array from magic items after every sort as well
public static void main(String[] args)
  {
	  String[] list = new String[size];
	  file(list);
	  selectSort(list);
	  file(list);
	  insertSort(list);
	  file(list);
	  System.out.println("Merge sort comparisons:" + MergeSort.sort(list));
	  file(list);
	  System.out.println("Quick sort comparisons:" 
	  + QuickSort.sort(list, 0, list.length -1));
    
  }//main

\end{lstlisting}

\section{File Method}.
\begin{lstlisting}[language = java]
  //this is basically the same file as lab 1
  //all it does is read in the items from magic items.
 public static void file(String[] items)
  {
	  String fileName = null;
	  String line = null;
	    
	    try
	    { 
	      fileName = "magicitems.txt";
	     
	      File theFile = new File(fileName);   
	      
	      Scanner input = new Scanner(theFile);
	    		 
	      for (int i = 0; i<size; i++)
	      {
	    	  line = input.nextLine().toLowerCase();
	    	  items[i] = line;
	      }
	   
	      input.close();
	      keyboard.close();
	    }//try
	    
	    catch(Exception ex)
	    {
	      System.out.println("Oops, something went wrong!");
	    }//catch
	    
	    //this sends back a message if something goes wrong in 
	    importing the text into the array from magic items
  }//file
  

\end{lstlisting}

\section{Selection Sort Method}
This method is the selection sort method. It runs through the whole array comparing the smallest item (aka earliest in the alphabet) with the other items and swaps if necessary and increments compare, basically it shifts which item is considered min and then once it runs through the whole array it will send it to the left of the array. The reason why it is has the run time of O(n to the 2nd) is because of the two for loops that go to n each that are nested. This is shown in the table below.
\begin{lstlisting}[language = java]
public static void selectSort(String[] items)
  {
	  int n = items.length;
	  int compare = 0;
	  for (int i = 0; i < n - 1; i++)
	  {
		  int smallPos = i;
		  for (int j = smallPos + 1; j < n; j++)
		  {
			  compare++;
			  if (items[j].compareTo(items[smallPos]) < 0)
			  {
				  smallPos = j;
			  }
		  }
		  String swap = items[i];
		  items[i] = items[smallPos];
		  items[smallPos] = swap;
	  }
	  System.out.println("Selection sort comparisons:" + compare);
  }//SS

\end{lstlisting}

\section{Insert Sort Method}.
What this method does looks to the left of of the index and compares it with that number and swaps if needed
The reason why it is has the run time of O(n to the 2nd) is because it has a while loop thats nested in a for loop that goes to n, and the while effectively goes to n because it runs through the whole array because of the for loop it's nested in. This is shown in the table below.
\begin{lstlisting}[language = java]
public static void insertSort(String[] items)
  {
	  int n = items.length;
	  int compare = 0;
	  
	  for (int i = 1; i < n ; i++)
	  {
		  String key = items[i];
		  int j = i - 1;
		  while ((j > -1) && (items[j].compareTo(key) > 0))
		  {
			  compare++;
			  items[j + 1] = items [j];
			  j--;
		  }
		  items[j+1] = key;
	  }
	  
	  System.out.println("Insert sort comparisons:" + compare);
  }//IS

\end{lstlisting}

\section{Merge Sort Class}
This method was inspired from a lot of psudocode from the text book and online sources and was easily the hardest one. Basically the first half is the divide part hence why you have many lines that say "/2", and they divide into 2 arrays called left and right. After that it will run through the array checking if the left is smaller than the right and if so make the value in the array that item. And that also means by process of elimination, if it isn't smaller then the right is smaller and hence that item should go in the array first. It's runtime is big o of n log n because the loops are going to the sub array of left and right as well as its being divided by 2. This is shown in the table below.
\begin{lstlisting}[language = java]
public class MergeSort 
{
	static int compare = 0;
	public static int sort(String[] items)
	{
		if (items.length > 1)
		{
			String[] left = new String[items.length / 2];
			String[] right = 
			new String[items.length - items.length / 2];
			
			for (int i = 0; i < left.length; i++) 
			{
                left[i] = items[i];
            }

            for (int i = 0; i < right.length; i++) 
            {
                right[i] = items[i + items.length / 2];
                compare++;
            }

            sort(left);
            sort(right);
            merge(items, left, right);
		}
		return compare;
	}//sort
	
	public static void merge(String[] items, String[] left, String[] right)
	{
		int i = 0;
		int j = 0;
		
		for (int k = 0; k < items.length; k++)
		{
			if (j >= right.length || (i < left.length && 
			left[i].compareToIgnoreCase(right[j]) < 0))
			{
				items[k] = left[i];
				i++;
				compare++;
			}
			else
			{
				items[k] = right[j];
				j++;
				compare++;
			}
		}
	}//merge
}


\end{lstlisting}

\section{Quick Sort Method}
This method was mostly from the textbook, basically you pass it a array, the starting num, and the end number so it knows how many times to go through the array. Which looking back I could of used a for loop for... whooops. Anyways the pivot number is going to start at whatever the end number is when you pass it in the sort method. Once the pivot is set, it will start to run through and compare the j (number in the loop) to the pivot and swap when needed until there is a array with a length of one. It's runtime is big o of n log n because it's being split apart by each pivot and having to work with multiple arrays on the left and right. This is shown in the table below.
\begin{lstlisting}[language = java]
public class QuickSort 
{
	static int compare = 0;
	
	public static int sort(String[] items, int begin, int end)
	{
		if (begin < end)
		{
			int partitionInd = partition(items, begin, end);
			
			sort(items, begin, partitionInd - 1);
			sort(items, partitionInd + 1, end);
		}
		
		return compare;
	}
	
	public static int partition(String[] array, int begin, int end)
	{
		String pivot = array[end];
		int i = (begin - 1);
		
		for (int j = begin; j < end; j++)
		{
			if (array[j].compareTo(pivot) < 0)
			{
				i++;
				String swap = array[i];
				array[i] = array[j];
				array[j] = swap;
			}
			compare++;
		}
		
		String swapTwo = array[i + 1];
		array[i+1] = array[end];
		array[end] = swapTwo;
		
		return i + 1;
	}//partition
}

\end{lstlisting}

\begin{center}
\begin{tabular}{ |c|c|c| } 
 \hline
 Sort Name & Complexity & number of compares \\ 
 Selection & O(n to the second) & 221445 \\ 
 Insert & O(n to the second) & 67906  \\ 
 Merge & O(n * log n) & 9615 \\ 
 Quick & O(n * log n) & 7238 \\ 
 \hline
\end{tabular}
\end{center}

\end{document}